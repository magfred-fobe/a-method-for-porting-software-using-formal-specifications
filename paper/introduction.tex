\section{Introduction}\label{introduction}

The computer software world is under constant development. New demands on a system could make it necessary to adapt it into a new environment than what it was originally designed for - a process called porting. 

\subsection{Background}

When porting software, one of the main concerns is that the functionality of the port adheres to the functionalities of the target source. The traditional way of verifying the behavior of computer software has been through the practice of testing. However, testing has limitations, and it is difficult to fully describe a system using testing alone \cite{testingisnecessary}. One way to complement the testing process to improve on this situation could be to set up a model of the system through a formal specification; a mathematically based technique with which the design of a system can be analyzed, and its functionalities be described with the purpose of helping with the implementation of systems and software. This means that independent of what language a software has been written in, if its behavior follows the formal specification, it provides functionality equivalent to any other software conforming to that same specification.

\subsection{Commissioned Work}
The customer for this project is Young Aces By Sylog (YABS); an IT consultant firm with one of its offices based in Stockholm, Sweden. As consultants they get hired to do work not normally part of their customers daily operations. A lot of this work concerns updating and porting system functionalities consisting of so-called legacy code. This is normally outdated code often severely limiting the performance of the system in question.

\subsection{Problem Specification}
The expressed problem that makes up the basis for this thesis is that the methods commonly used today are insufficient to verify the outcome of ported functionalities reliably. This thesis aims to answer the question: 

\textbf{is there a way to complement the current porting methods, which could increase the certainty that the ported code adheres to the original using a formal specification?}


\subsection{Purpose}

The use of formal specifications has indicated that it reduces the number of errors during software development. This practice has shown to create more reliable software while at the same time reducing production costs \cite{AMAZONFORMALSPEC}. The purpose of this thesis is to evaluate if formal specifications have the potential to provide a higher level of certainty that ported code adheres to the original. 

\subsection{Goal}
 
The goal of this thesis is to create a step-by-step methodology for porting software which uses a formal specification. Principally, to find whether the approach taken to incorporate a formal specification into the porting process yielded a valid result. A subjective evaluation of the strengths and weaknesses found when applying the methodology to the sample project will be made. 


\subsection{Target Audience}
The intended target audience is mainly practicing computer engineers as well as students of computer science interested in formal specifications and how it could be used in practice. It could also have the potential to be used as a basis for further work in the academic world.  

\subsection{Research Methodology}
The thesis will be performed as a qualitative case study where a detailed model for the methodology is described and then applied to a sample case. A method for evaluating the results will be described as part of the methodology and will be used to validate whether the process was successful. The quality of work and results of this thesis will be evaluated according to a set of quality criteria for qualitative research \cite{QUALITATIVE}.

\subsection{Delimitations}
The investigation's primary focus is on the feasibility of using formal specifications during a porting process; if it has the potential to be used to create a valid port of a system in a practical way by computer engineers during porting of computer software. The thesis will evaluate the validity of the port considering that the functionality provided is correct, but not whether it is achieved following any constraints on its performance.

The project is undertaken in a restricted time frame. Therefore, the scope of the source to which the methodology is applied is limited. This means that the efficiency of the method on other types of projects is left up to further investigation.
